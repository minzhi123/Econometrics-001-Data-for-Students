\documentclass[12pt]{article}
\usepackage[margin=1in]{geometry}
\setlength{\parindent}{0in}

%These packages allow the most of the common "mathly things"
\usepackage{amsmath,amsthm,amssymb}

%This package allows you to add graphs or any other images.
\usepackage{graphicx}

\usepackage{color}
\usepackage{enumerate}
\usepackage{multicol}

%------------------------------------------
% The stuff you want to edit starts here.
%------------------------------------------
\begin{document}

\title{Problem Set 3 Solutions} 
\author{Minzhi Meng}
\maketitle


%question1
{\bf Question 1} 
\bigskip\\
To capture the life-cycle pattern of wages, consider the following multiple linear regression model
$$wage = \beta_0 + \beta_1educ + \beta_2exper + \beta_3exper^2 + \epsilon $$\\
where $wage$ is the wage rate measured in dollar per hour, $educ$ is years of education, $exper$ is years of work experience, $exper^2$ is squared years of experience (in the dataset is called $expersq$). 

\begin{enumerate}[(1)]

%--------------------------------------------------------------------------------
%part(1)
\item Estimate the equation and report the results. (For full credit give your answer as a formula and a word explanation)
\smallskip\\
{\bf Solution:\\}
$wage = -3.9649 + 0.5953educ + 0.2683exper – 0.0046exper^2$\\
Standard interpretation. For example take the coefficient on $educ$. The estimated coefficient for $educ$ ($\beta_{1}$) is 0.5953. $p value < 0.05$, thus at 0.05 level of significance, the coefficient $\beta_1$ is statistically significant. It says that on average, one extra year of education increases the wage of a person by 0.5963 dollar per hour, holding other variables constant.

%--------------------------------------------------------------------------------
%part(2)
\item What is the marginal effect of a year increase in the work experience for a person with 18 years of work experience?
\smallskip\\
{\bf Solution:}
\smallskip\\
$\frac{\partial wage}{\partial exper}|_{educ}  = 0.2683 - 2 * 0.0046 * exper = 0.2683 - 2 * 0.0046 * 18 = 0.1027$


\end{enumerate}




\bigskip\\
\bigskip\\
%question2
{\bf Question 2}
\bigskip\\
Load data via link below and store in dataframe $fare$, where $year$ is 1997, 1998, 1999, 2000, $id$ is route identifier, $dist$ is the distance of the route measured in miles, $passen$ is the average number of passengers per day, $fare$ is the average one-way fare of the route measured in dollar per day, $conc$ is the percent of market controlled by the biggest carrier on route

\begin{enumerate}[(1)]
\item Generate a new variable called $yr00$ which takes on values 1 if the observation is in 2000 and 0 otherwise (a dummy variable).
\smallskip\\
{\bf Solution:\\}
skipped...


\item Estimate the model 1 using OLS and print the summary results.
\smallskip\\
{\bf Solution:\\}
Model 1):  $fare = 46.1894 + 0.0888dist+ 0.7336conc + \epsilon$



\item Consider the null hypothesis that $\beta_2 = 0$ and $\beta_3 = 0$ in Model 1. What is the alternative hypothesis? Conduct the test and state your decision. 
\smallskip\\
{\bf Solution:\\} 
$F( 2, 4593) = 1633$\\
$Prob > F = 0.0000$\\
The alternative hypothesis is that either of the coefficients is nonzero. The $p$ value is very low so we reject the null in favor of the alternative implying that these coefficients are nonzero.



\item State in words and numbers how we should interpret the estimated coefficients in Model 1. (for full credit explain how you know if it is statistically significant) 
\smallskip\\
{\bf Solution:\\} 
Standard interpretation. For example take the coefficient on $conc$. $\beta_3$ = 0.7336, $p value < 0.05$, thus at 0.05 level of significance, the coefficient for $conc$ ($\beta_3$) is statistically significant. It refers to market concentration and thus on average, an 1\% increase in the market concentration increases the fare by 0.7336 dollar, holding other variables constant.


\item Generate an interaction term called $yr00dist$ by multiplying the variable $dist$ by the variable you created in part (1), $yr00dist = yr00 * dist$
\smallskip\\
{\bf Solution:\\}
skipped...


\item Estimate the model 2 using OLS and print the summary results.
\smallskip\\
{\bf Solution:\\}
Model 2):  $fare = 41.6998 + 0.0893dist + 0.7438conc + 14.7778yr00 - 0.0016yr00dist + \epsilon$


\item What is the base group for Model 2?
\smallskip\\
{\bf Solution:\\} 
Where the dummy variables take the 0 values - in this case, where yr00 = 0, that is in all other years than 2000



\item State in words and numbers how we should interpret the estimated coefficients respectively in Model 2. (for full credit explain how you know if it is statistically significant)
\smallskip\\
{\bf Solution:\\} 
The coefficient of yr00dist is -0.0016. We can interpret that on average fare in 2000 decreases by 0.1\% less than it does in other years than 2000 when distance increases by one mile, holding other variables constant. However, the $p$ value is 0.604 which is larger than 0.05, thus at the level of 0.05 significance, the interaction term is not statistically significant. 




\end{enumerate}


\end{document}